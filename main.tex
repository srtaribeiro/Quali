\documentclass{beamer}
%
% Choose how your presentation looks.
%
% For more themes, color themes and font themes, see:
% http://deic.uab.es/~iblanes/beamer_gallery/index_by_theme.html
%
\mode<presentation>
{
  \usetheme{default}
  \usecolortheme{dove}
  \usefonttheme{structurebold}
  \setbeamertemplate{navigation symbols}{}
  \setbeamertemplate{caption}[numbered]
  \setbeamertemplate{itemize items}[circle]
}

\usepackage[brazil]{babel}
\usepackage[utf8]{inputenc}
\usepackage[natbibapa]{apacite}

\setbeamertemplate{frametitle}[default][center]

\title[Modelo de Apresentação]{Predição e Diagnóstico de Desempenho em Módulos de Usinas Solares Fotovoltaicas}
\author{Kymberlim Giovanna Martins Ribeiro\\
{\footnotesize \texttt{kymberlim.ribeiro@aluno.ufms.br}}\\{Orientador: Prof. Dr. Ricardo Ribeiro dos Santos}}
\institute{Mestrado Acadêmico em Ciência da Computação\\
Faculdade de Computação - UFMS}
\date{28 de Agosto de 2018}

\titlegraphic{
\includegraphics[height=1cm]{figures/capa/lscadlogo.png}
\hspace{1cm}
\includegraphics[height=1cm]{figures/capa/cec.jpg}
\hspace{1.5cm}
\includegraphics[height=1.25cm]{figures/capa/ufms_logo.png}
\hspace{1cm}
\includegraphics[height=1.25cm]{figures/capa/grafo_facom.png}
\hspace{1cm}
\includegraphics[height=1.25cm]{figures/capa/cem.jpg}
\hspace{1cm}
\includegraphics[height=1.25cm]{figures/capa/cep.jpg}
}

\begin{document}
\renewcommand{\BBAB}{\&}

\begin{frame}
  \titlepage
\end{frame}

\logo{\includegraphics[height=1cm]{figures/capa/lscadlogo.png}}

\begin{frame}{Sumário}
  \tableofcontents
\end{frame}

\section{Introdução}
\section{Sujidade}
\section{Trabalhos Relacionados}
\section{Proposta de Trabalho}

\begin{frame}[allowframebreaks]
  \frametitle{Referências}   
  \bibliographystyle{apacite}
  \bibliography{ref}
\end{frame}


%\begin{frame}{Introduction}

%\begin{itemize}
%  \item Your introduction goes here!
%  \item Use \texttt{itemize} to organize your main points.
%\end{itemize}

%\vskip 1cm

%\begin{block}{Examples}
%Some examples of commonly used commands and features are included, to help you get started.
%\end{block}

%\end{frame}

%\section{Some \LaTeX{} Examples}

%\subsection{Tables and Figures}

%\begin{frame}{Tables and Figures}

%\begin{itemize}
%\item Use \texttt{tabular} for basic tables --- see Table~\ref{tab:widgets}, for example.
%\item You can upload a figure (JPEG, PNG or PDF) using the files menu. 
%\item To include it in your document, use the \texttt{includegraphics} command (see the comment below in the source code).
%\end{itemize}

% Commands to include a figure:
%\begin{figure}
%\includegraphics[width=\textwidth]{your-figure's-file-name}
%\caption{\label{fig:your-figure}Caption goes here.}
%\end{figure}

%\begin{table}
%\centering
%\begin{tabular}{l|r}
%Item & Quantity \\\hline
%Widgets & 42 \\
%Gadgets & 13
%\end{tabular}
%\caption{\label{tab:widgets}An example table.}
%\end{table}

%\end{frame}

%\subsection{Mathematics}

%\begin{frame}{Readable Mathematics}

%Let $X_1, X_2, \ldots, X_n$ be a sequence of independent and identically distributed random variables with $\text{E}[X_i] = \mu$ and $\text{Var}[X_i] = \sigma^2 < \infty$, and let
%$$S_n = \frac{X_1 + X_2 + \cdots + X_n}{n}
%      = \frac{1}{n}\sum_{i}^{n} X_i$$
%denote their mean. Then as $n$ approaches infinity, the random variables %$\sqrt{n}(S_n - \mu)$ converge in distribution to a normal %$\mathcal{N}(0, \sigma^2)$.

%\end{frame}

\end{document}
